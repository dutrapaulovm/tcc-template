%------------------------------------------------------------
%------------    Estrutura do texto   -----------------------         

% Pacotes Básicos:
%\usepackage{lmodern}			    % Usa a fonte Latin Modern			

\usepackage[T1]{fontenc}		  % Selecao de codigos de fonte.
\usepackage[utf8]{inputenc}		% Codificacao do documento (conversão automática dos acentos)
\usepackage{mathptmx,helvet,courier}

%\usepackage{pslatex}                % Usa a fonte Times New Roman
\usepackage{lastpage}			    % Usado pela Ficha catalográfica
\usepackage{indentfirst}		  % Indenta o primeiro parágrafo de cada seção.
\usepackage[table]{xcolor}
\usepackage{color}				    % Controle das cores
\usepackage{graphicx}			    % Inclusão de gráficos
\usepackage{microtype} 		  	% para melhorias de justificação

\usepackage{tablefootnote} % para colocar footnotes em tabelas e figuras

% Pacotes Extras:
\usepackage[final]{pdfpages}
\usepackage{amsmath,amsthm}   %Símbolos Matemáticos
\usepackage{indentfirst} % Indenta primeiro parágrafo 
\usepackage[portuguese, ruled, linesnumbered,commentsnumbered, algo2e, vlined, lined, boxed, algochapter]{algorithm2e} % 
\usepackage{algorithm}
\usepackage{algpseudocode}


\usepackage{hyperref}
\usepackage{lineno,hyperref}
\usepackage[brazilian,hyperpageref]{backref}	 % Paginas com as citações na bibliograficas

%\usepackage[num]{abntex2cite}	% Citações padrão ABNT númerico
\usepackage[alf,abnt-emphasize=bf,abnt-full-initials=no,abnt-etal-list=5,abnt-etal-text=emph,abnt-repeated-author-omit=yes]{abntex2cite}
%\usepackage[alf,abnt-emphasize=bf,abnt-etal-list=0,abnt-etal-text=emph]{abntex2cite}

% se desejar justificar as referências (pela abnt é alinhado a esquerda)
% \usepackage[alf,abnt-emphasize=bf,bibjustif]{abntex2cite}

\usepackage{float} % para ajustar a posição das imagens

\usepackage{tikzsymbols} %para caracteres emoji
\usepackage{stackengine}
\usepackage{scalerel}
\usepackage{lscape}
\newcommand\dangersign[1][2ex]{%
  \renewcommand\stacktype{L}%
  \scaleto{\stackon[1.3pt]{\color{red}$\triangle$}{\tiny\bfseries !}}{#1}%
}

\usepackage{etoolbox}
%\usepackage[num]{abntex2cite}  % Citações numéricas

\usepackage{lipsum}

%para usar os algoritmos e matematica
\newcommand{\var}{\texttt}

\usepackage{amssymb}
\usepackage{amsmath}



% Defininfo Cores:
\definecolor{blue}{RGB}{25,25,112}
\definecolor{midgray}{gray}{.7}

\makeatletter % informações do PDF
\hypersetup{ % pagebackref=true,
	pdftitle={\@title}, 
	pdfauthor={\@author},
    pdfsubject={\imprimirpreambulo},
	pdfcreator={LaTeX with abnTeX2},
	pdfkeywords={abnt}{latex}{abntex}{abntex2}{trabalho acadêmico}, 
	colorlinks=true,     % false: boxed links; true: colored links
    linkcolor=black,          	% color of internal links
    citecolor=black,        		% color of links to bibliography
    filecolor=magenta,      	% color of file links
    urlcolor=black,
	bookmarksdepth=4 }
\makeatother
 

% -------------------------------------------- 
% Espaçamentos entre linhas e parágrafos 
\setlength{\parindent}{1.3cm} % O tamanho do parágrafo

% Controle do espaçamento entre um parágrafo e outro:
\setlength{\parskip}{0.2cm}  % tente também \onelineskip

% Definição de ambientes matemáticos em português 
\newtheorem{teorema}{Teorema}[chapter]
\newtheorem{axioma}{Axioma}[chapter]
\newtheorem{corolario}{Corolário}[chapter]
\newtheorem{lema}{Lema}[chapter]
\newtheorem{proposicao}{Proposição}[chapter]
\newtheorem{definicao}{Definição}[chapter]
\newtheorem{exemplo}{Exemplo}[chapter]
\newtheorem{observacao}{Observação}[chapter]

% Novos Comandos
\usepackage{tgtermes}
\renewcommand{\ABNTEXchapterfont}{\rmfamily\bfseries}


\providecommand{\imprimirnomemes}{}
\newcommand{\nomemes}[1]{\renewcommand{\imprimirnomemes}{#1}}

% Variáveis adicionais
\providecommand{\imprimirautorcite}{}
\newcommand{\autorcite}[1]{\renewcommand{\imprimirautorcite}{#1}} 

\providecommand{\imprimirsigla}{}
\newcommand{\sigla}[1]{\renewcommand{\imprimirsigla}{#1}}

\providecommand{\imprimircampus}{}
\newcommand{\campus}[1]{\renewcommand{\imprimircampus}{#1}}

\providecommand{\imprimiruf}{}
\newcommand{\uf}[1]{\renewcommand{\imprimiruf}{#1}}
\providecommand{\imprimircurso}{}
\newcommand{\curso}[1]{\renewcommand{\imprimircurso}{#1}}
\providecommand{\imprimirinstituto}{}
\newcommand{\instituto}[1]{\renewcommand{\imprimirinstituto}{#1}}
\providecommand{\imprimirdepartamento}{}
\newcommand{\departamento}[1]{\renewcommand{\imprimirdepartamento}{#1}}
\providecommand{\imprimirano}{}
\newcommand{\ano}[1]{\renewcommand{\imprimirano}{#1}}
\providecommand{\imprimirdia}{}
\newcommand{\dia}[1]{\renewcommand{\imprimirdia}{#1}}
\providecommand{\imprimirmes}{}
\newcommand{\mes}[1]{\renewcommand{\imprimirmes}{#1}}
\providecommand{\imprimirgrau}{}
\newcommand{\grau}[1]{\renewcommand{\imprimirgrau}{#1}}
\providecommand{\imprimirexaminadorum}{}
\newcommand{\examinadorum}[1]{
    \renewcommand{\imprimirexaminadorum}{#1}}
\providecommand{\imprimirexaminadordois}{}
\newcommand{\examinadordois}[1]{
    \renewcommand{\imprimirexaminadordois}{#1}}
\providecommand{\imprimirexaminadortres}{}
\newcommand{\examinadortres}[1]{
    \renewcommand{\imprimirexaminadortres}{#1}}
\providecommand{\imprimirexaminadorquatro}{}
\newcommand{\examinadorquatro}[1]{
    \renewcommand{\imprimirexaminadorquatro}{#1}}
\providecommand{\imprimirttorientador}{}
\newcommand{\ttorientador}[1]{
    \renewcommand{\imprimirttorientador}{#1}} 
\providecommand{\imprimirttcoorientador}{}
\newcommand{\ttcoorientador}[1]{
    \renewcommand{\imprimirttcoorientador}{#1}}
\providecommand{\imprimirttexaminadorum}{}
\newcommand{\ttexaminadorum}[1]{
    \renewcommand{\imprimirttexaminadorum}{#1}}
\providecommand{\imprimirttexaminadordois}{}
\newcommand{\ttexaminadordois}[1]{\renewcommand{
        \imprimirttexaminadordois}{#1}}
\providecommand{\imprimirttexaminadortres}{}
\newcommand{\ttexaminadortres}[1]{
    \renewcommand{\imprimirttexaminadortres}{#1}}
\providecommand{\imprimirttexaminadorquatro}{}
\newcommand{\ttexaminadorquatro}[1]{
    \renewcommand{\imprimirttexaminadorquatro}{#1}}


% Cria o comando \subtitulo. A norma define que o TITULO deve ser em caixa alta
% negrito, mas o subtitulo deve ser em caixa baixa.
\providecommand{\imprimirsubtitulo}{}
\newcommand{\subtitulo}[1]{\renewcommand{\imprimirsubtitulo}{#1}}

%----------------------------------------------------
\renewcommand{\imprimircapa}{  % Capa 
\begin{capa}
%\begin{center}\includegraphics[scale=1]{Figuras/ifmg.png}\end{center}
\begin{center}{
             \large \MakeTextUppercase{\imprimirinstituicao} - \MakeTextUppercase{\imprimirinstituto} \\
              %\imprimirdepartamento \\
              \MakeTextUppercase{\imprimircurso} \\
              \vspace{2cm}
			  \large {\imprimirautor} 
			  }\end{center}
\vfill
        \begin{center}
        \MakeTextUppercase{\large \textbf{\imprimirtitulo}}  \\
        \MakeTextLowercase{\textbf{\imprimirsubtitulo}}
				\vspace{2cm}
				%{\large \imprimirautor} 	
				\vfill
        {\large{\imprimirlocal~-~\imprimiruf \\ \imprimirano }}
        \end{center}
\end{capa}   } % Capa

%----------------------------------------------------
\renewcommand{\imprimirfolhaderosto}{% folha de rosto
    \begin{center}
    {{\MakeTextUppercase \imprimirautor}}  \\
		\vfill
		\large {\textbf{\MakeTextUppercase{\imprimirtitulo}}\\
        \MakeTextLowercase{\textbf{\imprimirsubtitulo}}}
    \end{center}
    \vfill 
    \begin{flushright} 
    \parbox{0.6\linewidth}{
		\imprimirtipotrabalho~apresentado ao \imprimirinstituto,~do \imprimirinstituicao,~como parte das exigências do curso de \imprimircurso~para a obtenção do título de \imprimirgrau. \\
		\vfill
		\textbf{\imprimirorientadorRotulo}~\imprimirorientador \\
		\vfill 
		\textbf{\imprimircoorientadorRotulo}~\imprimircoorientador}
   \end{flushright} 
   
	 \vfill
   \begin{center}
   {\large{\imprimirlocal~- \imprimiruf \\ \imprimirano}}
   \end{center} }  % folha de rosto

%----------------------------------------------------


%================================================================================
% Pacotes de citações
%================================================================================

% Configurações do pacote backref
% Texto padrão antes do número das páginas
\renewcommand{\backref}{}
% Define os textos da citação
\addto\captionsbrazil{% portugues-brasil
    % Usado sem a opção hyperpageref de backref
    %\renewcommand{\backrefpagesname}{Citado na(s) p{\'a}gina(s):~}
    \renewcommand*{\backrefalt}[4]{
    	\ifcase #1 %
    		%Nenhuma cita{\c c}{\~a}o no texto.%
    	\or
    		%Citado na p{\'a}gina #2.%
    	\else
    		%Citado #1 vezes nas p{\'a}ginas #2.%
    	\fi}%
}
\addto\captionsenglish{% ingles
    % Usado sem a opção hyperpageref de backref
    \renewcommand{\backrefpagesname}{Cited on page(s):~}
    \renewcommand*{\backrefalt}[4]{
    	\ifcase #1 %
    		No citation.%
    	\or
    		Cited on page #2.%
    	\else
    		Cited #1 times on pages #2.%
    	\fi}%
}
% ---
\makeatletter
\newcommand\thefontsize{Fonte tamanho: \f@size pt}
\makeatother



% para não deixar uma figura sozinha no centro de uma página,
% mas no topo da página.
\makeatletter
\setlength{\@fptop}{0pt}
\makeatother


%-----------------------------------------------------------
%-----------------------------------------------------------
\newcommand{\R}{\mathbb{R}}
\newcommand{\N}{\mathbb{N}}
\newcommand{\Z}{\mathbb{Z}}
\newcommand{\Q}{\mathbb{Q}}
\newcommand{\K}{\mathbb{K}}
\newcommand{\I}{\mathbb{I}}
\newcommand{\id}{\mathbf{1}}
\newcommand{\U}{\mathcal{U}}
\newcommand{\V}{{\cal V}}

\usepackage{minted}
\usepackage[portuguese, colorinlistoftodos, textsize=tiny]{todonotes}
\usepackage{setspace}
\usepackage{textcase}
\usepackage{amsfonts}
\usepackage{lscape}

\usepackage{lineno,hyperref}
\usepackage{caption}
\usepackage{subcaption}
\usepackage{multirow}
\usepackage{multicol}
\usepackage{bigstrut}
\usepackage{bm}
\usepackage{ragged2e} % Pacote para justificar o texto
\usepackage{datetime}
\usepackage{iflang}
\usepackage{ifthen}
\usepackage{xstring}
\usepackage{etoolbox}
\usepackage{atbegshi}
\usepackage{fancyhdr}

\newcommand{\removeSpaces}[1]{%
  \StrRemoveSpaces{#1}[\result]%
  \result
}

\newcommand{\Nota}[1]{\todo[inline]{#1}}
\newcommand{\NotaOK}[1]{\todo[inline]{Corrigido}}
\newcommand{\rev}[1]{\textcolor{red}{\textbf #1}}


\newcommand{\monthnamept}[1]{%
    \selectlanguage{portuguese}%
    \ifcase#1
        \or Janeiro%
        \or Fevereiro%
        \or Março%
        \or Abril%
        \or Maio%
        \or Junho%
        \or Julho%
        \or Agosto%
        \or Setembro%
        \or Outubro%
        \or Novembro%
        \or Dezembro%
    \fi
}

\newcommand{\monthnameen}[1]{%
    \selectlanguage{english}%
    \ifcase#1
        \or January%
        \or February%
        \or March%
        \or April%
        \or May%
        \or June%
        \or July%
        \or August%
        \or September%
        \or October%
        \or November%
        \or December%
    \fi
}



%--