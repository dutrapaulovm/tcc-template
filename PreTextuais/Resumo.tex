%--------------------------------------------------------------------------
%--------------------- Resumo em Português --------------------------------
%--------------------------------------------------------------------------

%\setlength{\absparsep}{18pt} % ajusta o espaçamento dos parágrafos do resumo
\begin{resumo}
O resumo é um pequeno texto onde o autor ressalta informações importantes sobre o
trabalho, como o objetivo, resultado, métodos utilizados e conclusão ou considerações finais.
O texto do mesmo precisa ser escrito de forma clara e objetiva, preferencialmente na terceira
pessoa do singular e em voz ativa, bem como deve conter entre 150 a 500 palavras e ser redigido com espaçamento de 1,5 entre linhas.
A palavra “RESUMO” é escrita em letras maiúsculas negritadas, centralizada na margem
superior da folha.
Após o resumo devem ser incluídas as palavras-chave. Recomenda-se a utilização de no
mínimo três e no máximo cinco palavras-chave que definam o assunto do trabalho, separadas
por ponto (.).

 \vspace{\onelineskip}
 \noindent
 \textbf{Palavras-chave}: Palavra-chave1. Palavra-chave2. Palavra-chave3. 
\end{resumo}

%--------------------------------------------------------------------------
%--------------------- Resumo em Inglês --------------------------------
%--------------------------------------------------------------------------
\begin{resumo}[\large ABSTRACT]
 \begin{otherlanguage*}{english}
   É a versão em língua estrangeira do resumo em língua portuguesa, com as mesmas características, para o idioma de divulgação internacional. O resumo em língua estrangeira deve ser apresentado em folha separada, seguido das palavras-chave, separadas por ponto (.). 
   O título deve ser escrito em letras maiúsculas negritadas, centralizada na margem superior da folha.
   No IFMG, optou-se pelo resumo no idioma inglês (Abstract) ou Espanhol (Resumen).


   \vspace{\onelineskip}
   \noindent 
   \textbf{Keywords}: Keywords1. Keywords2. Keywords3.
 \end{otherlanguage*}
\end{resumo}