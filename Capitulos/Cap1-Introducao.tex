\chapter{Introdução} \label{Introducao}

Este arquivo é o resultado do modelo-TCC-IFMG-ITR-2020-04.zip a ser editado em LaTeX, utilizando, por exemplo, editores online, como em: \url{https://pt.overleaf.com/}. O modelo, se seguido corretamente, atende às normas definidas no manual de Normalização de Trabalhos Acadêmicos do IFMG\footnote{Recomenda-se a leitura do manual na íntegra.} de fev/2020, que pode ser acessado na página: \url{https://tinyurl.com/y8aolnep}. 

Obs.: Apresente na Introdução uma síntese sobre o trabalho realizado, com apoio da literatura, situando a relevância do trabalho no contexto da sua área de formação e sua importância para o avanço do conhecimento. Neste capítulo também devem ser relatados os objetivos, a justificativa e a organização do trabalho dividindo em subseções. 

\dangersign[3ex] {\color{red}
Não se esqueça de modificar o número de páginas antes da introdução (sem contar a capa)!
No arquivo Monografia.tex altere a seguinte linha (no lugar de 14 coloque o valor contado de páginas): 
$\backslash setcounter\{page\}\{14\} $% Inserir aqui o número de páginas antes da introdução menos a capa
}

\section{Objetivos}

\subsection{\textit{Objetivo geral}}

O objetivo geral do trabalho é ...

\subsection{\textit{Objetivos específicos}}
\dangersign[3ex] {\color{red}
Lembre-se de colocar o comando /\~textit\{\} em subsections, para colocar em itálico o subtítulo, conforme normativa.
}

\section{Justificativa}

Justificativa para a realização deste trabalho.

\section{Organização do Texto}

Este trabalho está organizado da seguinte forma: (descrever)....

%Sugestões para estrutura da monografia:

% \begin{enumerate}[label=(\alph*)]
%   \item Introdução
%   \item Referencial Teórico (ou Revisão Bibliográfica)
%   \item Materiais e Métodos (ou Metodologia)
%   \item Resultados e Discussões
%   \item Conclusão (ou Considerações Finais)
%   \item Referências
% \end{enumerate}








