\chapter{Resultados} \label{resultado}

Neste capítulo são apresentados os resultados alcançados durante todo o trabalho, bem como uma discussão  e comparação com os resultados encontrados na literatura, destacando a importância desta pesquisa no contexto acadêmico.

A Equação \ref{eq:importante}
apresenta ....
%
\begin{equation}
\label{eq:importante}
    x^2      x^{2x} \cdot y + \vec{a}\times \vec{b} + \int_{a}^{b}x^3 dx = \xi_0 ,
\end{equation}
em que x é uma variável.

Na Equação~\ref{eq:transformada} observa-se que ...
%
\begin{equation}
\label{eq:transformada}
\vec{z} = \begin{bmatrix}
x_d 
\\ 
y_d 
\end{bmatrix} = \begin{bmatrix}
x_c + d\cos\theta
\\ 
y_c + d\sin\theta 
\end{bmatrix},
\end{equation}
em que $d$ é a distância até o centro do objeto.

Outro exemplo de expressão matemática:
%
\begin{equation}
   3\mu_0 \cdot \left \{ \frac{2}{3} \cdot \vec{x}\times \vec{y} \right \} \\
   \int_{x_1}^{x_2}hdx .
\end{equation}

Para as equações seguir também os modelos acima.

\section{Algoritmos} 
Códigos de programas não são consideradas figuras, desta forma, não necessitam de possuir legenda. Por exemplo, o código a seguir mostra um trecho de uma programa em Python.
\begin{minted}[mathescape,
               linenos,
               numbersep=5pt,
               gobble=2,
               frame=lines,
               framesep=2mm]{python}
    print("Hello World")

    for i in range(10):
        print(i)

\end{minted}
Algoritmos geralmente são apresentados como pseudo-códigos, os quais possuem uma formatação formal conhecida dos livros de computação. Diferentemente das listagens, os algoritmos costumam possuir legendas, como no algoritmo ~\ref{alg:exemplo} abaixo.
\begin{algorithm*}
\label{alg:exemplo}
 \caption{Ler número e imprimir se é par ou não.}

     \Entrada{número, ($numero$).}
     \Saida{Se o número é par ou não}
     \Inicio{
         \textbf{ler} $numero$\;
         \eSe {$numero \% 2 = 0$} {
             \textbf{imprimir} $numero$, " par"\;
         } {
         \textbf{imprimir} $numero$, " impar"\;
         }
        }
\end{algorithm*}

\section{Notas de rodapé}

Utilize as notas de rodapé\footnote{As notas devem ser alinhadas sendo que na segunda linha da mesma nota, a primeira letra deve estar abaixo da primeira letra da primeira palavra da linha superior, destacando assim o expoente.} para realizar esclarecimentos rápidos sobre algum conceito ou referência. Por exemplo, um endereço de uma página web. O IF Sudeste MG - Campus Muriaé\footnote{https://www.ifsudestemg.edu.br/muriae} é uma instituição.

\section{Tabelas}
Para criar uma tabela, siga o exemplo conforme da Tabela~\ref{tb:exemplo_tabela}.

\begin{table}[!ht]
     \caption{Exemplo tabela}\label{tb:exemplo_tabela}
	\centering
	{\footnotesize
    \begin{tabular}{|l|l|l|l|}
    \hline
        \textbf{Ordem} & \textbf{A} & \textbf{B} & \textbf{C} \\ \hline
        1 & 1 & 1 & 1 \\ \hline
        2 & 2 & 2 & 2 \\ \hline
        3 & 3 & 3 & 3 \\ \hline
    \end{tabular}
    }
\end{table}

Para facilitar a criação de tabelas em LateX, você pode utilizar os seguintes links: 
\begin{itemize}
\item \url{https://tableconvert.com/pt/latex-generator}
\item \url{https://www.tablesgenerator.com/}
\item \url{https://www.latex-tables.com/}
\item \url{https://clevert.com.br/latex/latextable.php}
\end{itemize}

Caso utilize o Excel, você pode utilizar o seguinte suplemento para converter planilhas em tabelas no formato LateX: \href{https://www.ctan.org/tex-archive/support/excel2latex/}{https://www.ctan.org/tex-archive/support/excel2latex/}