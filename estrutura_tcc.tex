%================================================================%
%=============  Modelo de Trabalho de Conclusão de curso ========%
%========================  IFMG =================================% 
% AUTORES:
% prof. Elias J R Freitas =======================================%
% ej-ensino.com.br
%
% baseado no modelo DEPRO-UFOP ==================================%
% estrutura elaborada inicialmente por
% Marcelus Xavier Oliveira  email: marcelusxavier@gmail.com =====%
% Dayanne Gouveia Coelho  email: dayagc@gmail.com ===============%

%================================================================%
% Proposta de texto em conformidade com normas da ABNT ----------%
% Modelo em conformidade com 
% Manual de Normalização de Trabalhos Acadêmicos 21/02/2020
% https://www.ifmg.edu.br/portal/ensino/bibliotecas/
% arquivos-bibliotecas/copy_of_ManualdeNormalizaoIFMG2020.pdf
% implementadas pelo projeto abntex2, que pode ser acessado pela %
% página  http://abntex2.googlecode.com/  -----------------------%
%================================================================%

%================================================================%
% Versao abntex2
%======================== Versão 2022/08 ========================%
%================================================================%


\documentclass[12pt, % tamanho da fonte
	%openright,	% capítulos começam em pág ímpar
	oneside, %para impressão apenas em um lado (formato digital).  		  
	% twoside, %para impressão em frente e verso.  
	a4paper,			% tamanho do papel. 
	english,			% Idioma adicional para hifenização
	brazil				% Idioma principal 
	]{packages/abntex2-ifmg}
	
\input{packages/packages}  % Estrutura do Texto e Pacotes Principais
%\selectlanguage{portuguese}
% -- Informações para Capa e Folha de Rosto a serem editadas

\titulo{Título do trabalho:}
\titulotrabalho{Primeiro Trabalho - 1° Semestre}
\titulodisciplina{CAPA PARA TRABALHOS SIMPLES}

% caso o trabalho não tenha subtítulo comentar linha abaixo e retirar dois pontos do título do trabalho
\subtitulo{Subtítulo do trabalho}

\autor{Nome Completo do Aluno} \autorcite{SOBRENOME, Nome}
\local{Muriaé} \uf{MG}
\data{31 de agosto de 2022} \dia{31} \mes{08} \ano{2022} %deixar sem preencher antes do TCC
\orientador{Prof. Dr. Nome do orientador}  % Nome do orientador 
\ttorientador{IFMG} % Instituição do orientador
\coorientador{Prof. Me. Nome do Coorientador}   % Nome do coorientador
\ttcoorientador{IFMG} % Instituição do Coorientador
\instituicao{Instituto Federal de Educação Ciência e Tecnologia de Minas Gerais} \sigla{IFSUDESTEMG}
\instituto{\textit{Campus} Muriaé}
%\departamento{Departamento de Engenharia de Controle e Automação e Técnicas Fundamentais}
\curso{Gestão da Tecnologia da Informação}	
\tipotrabalho{Trabalho de conclusão de curso}
\grau{Tecnólogo em Gestão da Tecnologia da Informação}

%------Nomes dos examinadores.  
\examinadorum{Prof. Me. Membro da Banca 1} \ttexaminadorum{UFXX}
\examinadordois{Prof. Dr. Membro da Banca  2} \ttexaminadordois{IFMG}
%\examinadortres{Prof. Dr. Membro da Banca  3} \ttexaminadortres{Universidade Federal de ... - UFXX}
%\examinadorquatro{Prof. Dr. Membro da Banca  4} \ttexaminadorquatro{Universidade Federal de ... - UFXX}





% ------------------------------------------------------
\makeindex   

\begin{document} % Início do documento


\frenchspacing  % Retira espaço obsoleto entre as frases.

% ----------------------------------------------------------
% -- Elementos Pré-Textuais: -------------------------------
\pagenumbering{roman}

%Comentar essa linha caso seja TCC
%\imprimircapatrabalho %Capa para trabalhos simples


\imprimircapa  % Capa
\imprimirfolhaderosto % Folha de rosto


% ---------------------------------------------------------------
% ----------------  Ficha Catalográfica  -------------------------
% ---------------------------------------------------------------
% Modelo de ficha catalográfica. Você deverá substituir esta
% folha na versão final da monografia por um pdf fornecido pela 
% biblioteca. Salve o modelo oficial como ficha_catalografica.pdf
% e use o comando abaixo para inseri-lo na versão final do texto.

%\begin{fichacatalografica}
%    \includepdf{ficha_catalografica.pdf}
%\end{fichacatalografica}



%% Modelo de Como fazer a Ficha Catalográfica:




\begin{fichacatalografica}
\sffamily
\vspace*{\fill}	
\begin{center}
\begin{minipage}[c][8cm]{13.5cm}
\begin{center}

\end{center}
\end{minipage}

\fbox{\hspace*{1cm} 
\begin{minipage}[c][7.5cm][t]{12.5cm}

\SingleSpacing \small
  \imprimirautorcite. \par
  \imprimirtitulo~/ \imprimirautor. - Muriaé, \imprimirano. \\ \par
  VIII, \pageref{LastPage} p. 29,7cm \\
  [12pt]
  \imprimirorientadorRotulo~\imprimirorientador\\ %\par
   % {\IfNonempty{\inserecoorientador}{%
    %   {\hspace*{0.45cm} \inserecoorientadorOU ~\inserecoorientador \par }%
    %}
   %}
  \imprimirtipotrabalho~-~\imprimirinstituicao.~\imprimirinstituto,~\imprimirano. \\[12pt]
  \imprimirpalavraschavecat
  \vspace*{12pt}
  \hspace*{0.3cm}
\end{minipage} \hspace*{0.3cm}}
\end{center}

\end{fichacatalografica}
\include{PreTextuais/FichaAprovacao} 
%\include{PreTextuais/Dedicatoria}
%\include{PreTextuais/Agradecimento}
%\include{PreTextuais/Epigrafe}
\include{PreTextuais/Resumo}

\renewcommand{\listfigurename}{\large LISTA DE ILUSTRA\c{C}\~{O}ES}
\pdfbookmark[0]{\listfigurename}{lof}
\listoffigures*   % Cria a Lista de Figuras
\cleardoublepage

% inserir lista de quadros
% ---
%\pdfbookmark[0]{\listofquadrosname}{loq}
%\listofquadros*
%\cleardoublepage
% ---

\pdfbookmark[0]{\listtablename}{lot}
\listoftables*  % Cria a lista de Tabelas
\cleardoublepage

%\renewcommand{\listalgorithmcfname}{Lista de algoritmos}
%\pdfbookmark[0]{\listalgorithmcfname}{lof}
%\listofalgorithmes   % Cria a lista de Tabelas
%\cleardoublepage

\include{PreTextuais/ListaSiglas}
\include{PreTextuais/ListaSimbolos}


\pdfbookmark[0]{\contentsname}{toc}
\tableofcontents*

\cleardoublepage




% ----------------------------------------------------------
% -- Capítulos do Trabalho: --------------------------------
\pagenumbering{arabic} 
\textual 

%%%%%%%%%%%% ORDEM DOS ARQUIVOS %%%%%%%%%%%%%%%%%%%

\setcounter{page}{10} % Inserir aqui o número de páginas antes da introdução menos a capa
\chapter{Introdução} \label{Introducao}

Este arquivo é o modelo a ser editado em LaTeX, utilizando, por exemplo, editores online, como em: \url{https://pt.overleaf.com/}. O modelo, se seguido corretamente, atende às normas definidas no manual de Normalização de Trabalhos Acadêmicos do IF SUDESTE MG\footnote{Recomenda-se a leitura do manual na íntegra.}.%, que pode ser acessado na página: \url{https://tinyurl.com/y8aolnep}. 

Obs.: Apresente na Introdução uma síntese sobre o trabalho realizado, com apoio da literatura, situando a relevância do trabalho no contexto da sua área de formação e sua importância para o avanço do conhecimento. Neste capítulo também devem ser relatados os objetivos, a justificativa e a organização do trabalho dividindo em subseções~\cite{loboguia}. 

%\dangersign[3ex] {\color{red}
%Não se esqueça de modificar o número de páginas antes da introdução (sem contar a capa)!
%No arquivo estrutura\_tcc.tex altere a seguinte linha (no lugar de 10 coloque o valor contado de páginas): 
%$\backslash setcounter\{page\}\{6\} $% Inserir aqui o número de páginas antes da introdução menos a capa
%}
\section{Objetivos}

\subsection{\textit{Objetivo geral}}

O objetivo geral do trabalho é ...

\subsection{\textit{Objetivos específicos}}
\dangersign[3ex] {\color{red}
Lembre-se de colocar o comando /\~textit\{\} em subsections, para colocar em itálico o subtítulo, conforme normativa.
}

\section{Justificativa}

Justificativa para a realização deste trabalho.

\section{Organização do Texto}

Este trabalho está organizado da seguinte forma: (descrever)....

%Sugestões para estrutura da monografia:

% \begin{enumerate}[label=(\alph*)]
%   \item Introdução
%   \item Referencial Teórico (ou Revisão Bibliográfica)
%   \item Materiais e Métodos (ou Metodologia)
%   \item Resultados e Discussões
%   \item Conclusão (ou Considerações Finais)
%   \item Referências
% \end{enumerate}









%\include{Capitulos/Cap2-RevisaoBibliografica}
\include{Capitulos/Cap3-Metodologia}
\chapter{Resultados} \label{resultado}

Neste capítulo são apresentados os resultados alcançados durante todo o trabalho, bem como uma discussão  e comparação com os resultados encontrados na literatura, destacando a importância desta pesquisa no contexto acadêmico.

A Equação \ref{eq:importante}
apresenta ....
%
\begin{equation}
\label{eq:importante}
    x^2      x^{2x} \cdot y + \vec{a}\times \vec{b} + \int_{a}^{b}x^3 dx = \xi_0 ,
\end{equation}
em que x é uma variável.

Na Equação~\ref{eq:transformada} observa-se que ...
%
\begin{equation}
\label{eq:transformada}
\vec{z} = \begin{bmatrix}
x_d 
\\ 
y_d 
\end{bmatrix} = \begin{bmatrix}
x_c + d\cos\theta
\\ 
y_c + d\sin\theta 
\end{bmatrix},
\end{equation}
em que $d$ é a distância até o centro do objeto.

Outro exemplo de expressão matemática:
%
\begin{equation}
   3\mu_0 \cdot \left \{ \frac{2}{3} \cdot \vec{x}\times \vec{y} \right \} \\
   \int_{x_1}^{x_2}hdx .
\end{equation}

Para as equações seguir também os modelos acima.

\section{Algoritmos} 
Códigos de programas geralmente não são consideradas como figuras, desta forma, não necessitam de possuir legenda. Por exemplo, o código a seguir mostra um trecho de uma programa em Python.
\begin{minted}[mathescape,
               linenos,
               numbersep=5pt,
               gobble=2,
               frame=lines,
               framesep=2mm]{python}
    print("Hello World")

    for i in range(10):
        print(i)
\end{minted}

Caso seja necessário considerar código como figuras, siga o exemplo da Figura~\ref{fig:codigo_python}.

\begin{figure}[H] % h! para here, b para bottom e t para top
\centering   
\caption{Exemplo de código em Python.}    
\begin{minted}[mathescape,
               linenos,
               numbersep=5pt,
               gobble=2,
               frame=lines,
               framesep=2mm]{python}
    print("Hello World")

    for i in range(10):
        print(i)
\end{minted}
\label{fig:codigo_python}
\end{figure}

Algoritmos geralmente são apresentados como pseudo-códigos, os quais possuem uma formatação formal conhecida dos livros de computação. Diferentemente das listagens, os algoritmos costumam possuir legendas, como no algoritmo~\ref{alg:exemplo} abaixo.
\begin{algorithm*}
 \caption{Ler número e imprimir se é par ou não.}
 \Entrada{número, ($numero$).}
 \Saida{Se o número é par ou não}
 \Inicio{
     \textbf{ler} $numero$\;
     \eSe {$numero \% 2 = 0$} {
         \textbf{imprimir} $numero$, " par"\;
     } {
     \textbf{imprimir} $numero$, " impar"\;
     }
    }
    \label{alg:exemplo}
\end{algorithm*}

\section{Notas de rodapé}

Utilize as notas de rodapé\footnote{As notas devem ser alinhadas sendo que na segunda linha da mesma nota, a primeira letra deve estar abaixo da primeira letra da primeira palavra da linha superior, destacando assim o expoente.} para realizar esclarecimentos rápidos sobre algum conceito ou referência. Por exemplo, um endereço de uma página web. O IF Sudeste MG - Campus Muriaé\footnote{https://www.ifsudestemg.edu.br/muriae} é uma instituição.

\section{Tabelas}
Para criar uma tabela, siga o exemplo conforme da Tabela~\ref{tb:exemplo_tabela}.

\begin{table}[!ht]
     \caption{Exemplo tabela}\label{tb:exemplo_tabela}
	\centering
	{\footnotesize
    \begin{tabular}{|l|l|l|l|}
    \hline
        \textbf{Ordem} & \textbf{A} & \textbf{B} & \textbf{C} \\ \hline
        1 & 1 & 1 & 1 \\ \hline
        2 & 2 & 2 & 2 \\ \hline
        3 & 3 & 3 & 3 \\ \hline
    \end{tabular}
    }
\end{table}

Para facilitar a criação de tabelas em LateX, você pode utilizar os seguintes links: 
\begin{itemize}
\item \url{https://tableconvert.com/pt/latex-generator}
\item \url{https://www.tablesgenerator.com/}
\item \url{https://www.latex-tables.com/}
\item \url{https://clevert.com.br/latex/latextable.php}
\end{itemize}

Caso utilize o Excel, você pode utilizar o seguinte suplemento para converter planilhas em tabelas no formato LateX: \href{https://www.ctan.org/tex-archive/support/excel2latex/}{https://www.ctan.org/tex-archive/support/excel2latex/}
\include{Capitulos/Cap5-Conclusao}


% ----------------------------------------------------------
% -- Elementos Pós-Textuais: -------------------------------
\postextual  
\bibliography{bibliografia} % Referências bibliográficas
\include{PosTextuais/Apendice}
\include{PosTextuais/Anexo}
%\includepdf[pages={1,3-5}]{PosTextuais/includepdfpages.pdf}
%\phantompart  \printindex  % Indice Remissivo
% ----------------------------------------------------------
\end{document}  % fim do documento
